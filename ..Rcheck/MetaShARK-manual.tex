\nonstopmode{}
\documentclass[letterpaper]{book}
\usepackage[times,inconsolata,hyper]{Rd}
\usepackage{makeidx}
\usepackage[utf8]{inputenc} % @SET ENCODING@
% \usepackage{graphicx} % @USE GRAPHICX@
\makeindex{}
\begin{document}
\chapter*{}
\begin{center}
{\textbf{\huge Package `MetaShARK'}}
\par\bigskip{\large \today}
\end{center}
\begin{description}
\raggedright{}
\inputencoding{utf8}
\item[Title]\AsIs{metashark}
\item[Version]\AsIs{0.0.0.9000}
\item[Author]\AsIs{Elie ARNAUD ([aut,cre])}
\item[Maintainer]\AsIs{Elie ARNAUD }\email{earnaud@outlook.com}\AsIs{}
\item[Description]\AsIs{MetaShARK is a Shiny application allowing its
user to fill metadata for a given ecological
dataset of his owns. The data description relies
on Ecological Metadata Language (EML).}
\item[License]\AsIs{MIT}
\item[Encoding]\AsIs{UTF-8}
\item[LazyData]\AsIs{true}
\item[Imports]\AsIs{attempt, cedarr, config, data.table, dataone, datapack, dplyr,
DT, EML, EMLassemblyline, emldown, fs, gdata, glue, golem,
htmltools, jsonlite, markdown, methods, mime, processx, RCurl,
readtext, RefManageR, rmarkdown, rorcid, shinyAce, shinyBS,
shinycssloaders, shinydashboard, shinyFiles, shinyjs,
shinydashboardPlus, shinyFeedback, shinyWidgets, shiny (>=
1.5.0), shinyTree, stringr, SummeRnote, textutils, tippy,
tagsinput, taxonomyCleanr, utils, xml2}
\item[Remotes]\AsIs{ThinkR-open/tagsinput, ropenscilabs/emldown}
\item[RoxygenNote]\AsIs{7.1.1}
\item[Roxygen]\AsIs{list(markdown = TRUE)}
\item[URL]\AsIs{}\url{https://github.com/earnaud/MetaShARK-v2.git}\AsIs{}
\item[BugReports]\AsIs{}\url{https://github.com/earnaud/MetaShARK-v2.git/issues}\AsIs{}
\item[Suggests]\AsIs{testthat (>= 3.0.0), knitr}
\item[VignetteBuilder]\AsIs{knitr}
\item[Depends]\AsIs{R (>= 2.10)}
\item[Collate]\AsIs{'zzz.R' 'utils\_writeText.R' 'utils\_URL\_Input.R'
'utils\_readPlaintext.R' 'utils\_readDataTable.R'
'utils\_reactiveTrigger.R' 'utils\_reactiveDirReader.R'
'utils\_listReactiveValues.R' 'utils\_isContentTruthy.R'
'utils\_insertModuleBox.R' 'utils\_followPath.R' 'utils\_devmsg.R'
'utils\_collapsible.R' 'utils\_additional\_HTML.R' 'utils\_grep.R'
'about\_module.R' 'about\_module\_functions.R'
'documentation\_module.R' 'eal\_1\_SelectDP.R' 'eal\_2\_DataFiles.R'
'eal\_2\_DataFiles\_helpers.R' 'eal\_3\_Attributes.R'
'eal\_3\_Attributes\_dates.R' 'eal\_3\_Attributes\_helpers.R'
'eal\_4\_CatVars.R' 'eal\_4\_CatVars\_helpers.R' 'eal\_5\_GeoCov.R'
'eal\_5\_GeoCov\_helpers.R' 'eal\_6\_TaxCov.R'
'eal\_7\_ORCID\_helpers.R' 'eal\_7\_Personnel.R'
'eal\_7\_Personnel\_helpers.R' 'eal\_7\_roleInput.R' 'eal\_8\_Misc.R'
'eal\_8\_Misc\_helpers.R' 'eal\_9\_MakeEML.R' 'eal\_pages.R'
'eal\_x\_Annotations.R' 'eal\_x\_Annotations\_helpers.R'
'fill\_module.R' 'fill\_module\_saves.R' 'fill\_pages.R' 'header.R'
'metafin.R' 'runMetashark.R' 'savevariable\_functions.R'
'settings\_rightSideBar.R' 'template\_functions.R'
'upload\_module.R' 'upload\_module\_functions.R'
'welcome\_module.R' 'ui.R' 'server.R'}
\item[Config/testthat/edition]\AsIs{3}
\end{description}
\Rdcontents{\R{} topics documented:}
\inputencoding{utf8}
\HeaderA{centered}{centered}{centered}
%
\begin{Description}\relax
Returns the content styled as centered text.
\end{Description}
%
\begin{Usage}
\begin{verbatim}
centered(...)
\end{verbatim}
\end{Usage}
%
\begin{Arguments}
\begin{ldescription}
\item[\code{...}] any HTML tag.
\end{ldescription}
\end{Arguments}
%
\begin{Value}
A div with \code{style = "text-align: center"}
\end{Value}
\inputencoding{utf8}
\HeaderA{checkFeedback}{checkFeedback relies on shinyFeedback to automate feedback on one input.}{checkFeedback}
%
\begin{Description}\relax
checkFeedback

relies on shinyFeedback to automate feedback on one input.
\end{Description}
%
\begin{Usage}
\begin{verbatim}
checkFeedback(
  input,
  id,
  condition = NULL,
  silent = FALSE,
  type = c("danger", "warning"),
  text = NULL
)
\end{verbatim}
\end{Usage}
%
\begin{Arguments}
\begin{ldescription}
\item[\code{id}] character. An ID string that corresponds with the ID used to call
the module's UI function.

\item[\code{condition}] logical. Determines if the feedback is positive or not.

\item[\code{silent}] logical. If TRUE, the only feedback occurs when \code{condition} is
met. If FALSE, also displays feedbacks for other cases.

\item[\code{type}] character. Either "danger" or "warning", sets the type of
feedback in case \code{condition} is not met.

\item[\code{text}] character. What message to display in case of unmet condition.
\end{ldescription}
\end{Arguments}
\inputencoding{utf8}
\HeaderA{collapsibleUI}{collapsibleUI}{collapsibleUI}
\aliasA{collapsible}{collapsibleUI}{collapsible}
%
\begin{Description}\relax
A shiny module to get a div collapsed by clicking on a link.
\end{Description}
%
\begin{Usage}
\begin{verbatim}
collapsibleUI(id, label, .hidden = TRUE, ..., class = NULL)

collapsible(id)
\end{verbatim}
\end{Usage}
%
\begin{Arguments}
\begin{ldescription}
\item[\code{id}] character. An ID string that corresponds with the ID used to call
the module's UI function.

\item[\code{label}] character. A label appearing on the clickable link.

\item[\code{.hidden}] logical. A flag to make the UI display as collapsed or not.

\item[\code{...}] shiny UI elements. Any UI element displayed as core content.

\item[\code{class}] character. CSS class to apply to ... .
\end{ldescription}
\end{Arguments}
%
\begin{Section}{Functions}
\begin{itemize}

\item{} \code{collapsible}: Server part for collapsible widget module.

\end{itemize}
\end{Section}
\inputencoding{utf8}
\HeaderA{followPath}{followPath}{followPath}
%
\begin{Description}\relax
Takes a hierarchy list (tree), a path written in a vector pasted
with sep, and returns the leaf.
\end{Description}
%
\begin{Usage}
\begin{verbatim}
followPath(tree, path, sep = "/", root = "root")
\end{verbatim}
\end{Usage}
%
\begin{Arguments}
\begin{ldescription}
\item[\code{tree}] explored hierarchy list thanks to path

\item[\code{path}] vector of characters matching tree's nodes' names and
separated with sep.

\item[\code{sep}] separators between path elements (aka tree names)

\item[\code{root}] if your path has a root name for root node, enter its name here.
Else, enter NULL.
\end{ldescription}
\end{Arguments}
\inputencoding{utf8}
\HeaderA{insertModuleBox}{insertModuleBox}{insertModuleBox}
%
\begin{Description}\relax
Convenience function to insert a container able to auto-destroy.
\end{Description}
%
\begin{Usage}
\begin{verbatim}
insertModuleBox(
  id,
  name = NULL,
  selector,
  moduleUI,
  moduleUI.args = list(),
  module,
  module.args = list()
)
\end{verbatim}
\end{Usage}
%
\begin{Arguments}
\begin{ldescription}
\item[\code{id}] namespaced id for the UI (previous modules namespaces will be
removed for server)

\item[\code{name}] name to display for the current container. If set to NULL (default),
\code{id} will be used. If set to FALSE, no name will be displayed.

\item[\code{selector}] A string that is accepted by jQuery's selector (i.e. the
string s to be placed in a \$(s) jQuery call). (See \code{shiny::insertUI})

\item[\code{moduleUI, moduleUI.args}] UI for the module to embed and its potential
arguments provided as a list (empty list by default).

\item[\code{module, module.args}] server for the module to embed and its potential
arguments provided as a list (empty list by default).

FIXME finish this module
\end{ldescription}
\end{Arguments}
\inputencoding{utf8}
\HeaderA{isContentTruthy}{Check for data structures validity}{isContentTruthy}
%
\begin{Description}\relax
Check if \code{x} is truthy (as shiny::isTruthy) or not, but also
checks for its potential content. This function uses \code{unlist()} and \code{all()} to
check every bit of the variable.
\end{Description}
%
\begin{Usage}
\begin{verbatim}
isContentTruthy(x)
\end{verbatim}
\end{Usage}
%
\begin{Arguments}
\begin{ldescription}
\item[\code{x}] argument to check fo truthiness
\end{ldescription}
\end{Arguments}
%
\begin{Value}
A logical indicating whether or not the variable is Truthy.
\end{Value}
\inputencoding{utf8}
\HeaderA{isHTMLTruthy}{Check for HTML tags validity}{isHTMLTruthy}
%
\begin{Description}\relax
Check if a HTML tag is well formed, according to opening and closing chevrons.
\end{Description}
%
\begin{Usage}
\begin{verbatim}
isHTMLTruthy(x)
\end{verbatim}
\end{Usage}
%
\begin{Arguments}
\begin{ldescription}
\item[\code{x}] HTML tag. Target of the evaluation.
\end{ldescription}
\end{Arguments}
%
\begin{Value}
A logical value indicating whether or not the tag is truthy.
\end{Value}
\inputencoding{utf8}
\HeaderA{listReactiveValues}{List Reactive Values}{listReactiveValues}
%
\begin{Description}\relax
Allows to turn a `reactiveValues`` object into a non-reactive list. Uses
recursive method.
\end{Description}
%
\begin{Usage}
\begin{verbatim}
listReactiveValues(rv, lv = 0, name = "root")
\end{verbatim}
\end{Usage}
%
\begin{Arguments}
\begin{ldescription}
\item[\code{rv}] reactiveValues. Target to turn into list.

\item[\code{lv}] internal. Verbose purposes.

\item[\code{name}] internal. Current root node name
\end{ldescription}
\end{Arguments}
\inputencoding{utf8}
\HeaderA{makeReactiveTrigger}{reactiveTrigger}{makeReactiveTrigger}
%
\begin{Description}\relax
Create a 'reactiveTrigger' object (NOT a proper R class) with two methods:
\begin{enumerate}

\item{} depend() : must be written in a code chunk to execute on triggering
\item{} trigger() : when executed, trigger the object (and all the "depending" code chunks)
This function is freely reused from Dean Attali's work: check it out, it is
\Rhref{https://github.com/daattali/advanced-shiny/tree/master/reactive-trigger}{awesome}

\end{enumerate}

\end{Description}
%
\begin{Usage}
\begin{verbatim}
makeReactiveTrigger(dev = FALSE, label = "")
\end{verbatim}
\end{Usage}
\inputencoding{utf8}
\HeaderA{optional}{optional}{optional}
%
\begin{Description}\relax
Fill optional arguments of a function if the given argument comes to be not
truthy (e.g. by getting it after a script execution and passing it to a
function).
\end{Description}
%
\begin{Usage}
\begin{verbatim}
optional(x, type = NULL)
\end{verbatim}
\end{Usage}
%
\begin{Arguments}
\begin{ldescription}
\item[\code{x}] value provided for the argument

\item[\code{type}] typed value (e.g. character()) in case option is invalid.
\end{ldescription}
\end{Arguments}
\inputencoding{utf8}
\HeaderA{readDataTable}{Read Data Tables}{readDataTable}
%
\begin{Description}\relax
Disclaimer: this function is still maturing. There is no guarantee about its
performance.
Guess the type of the data file (e.g. .xls* or not) and loads it accordingly.
\end{Description}
%
\begin{Usage}
\begin{verbatim}
readDataTable(file, data.table = FALSE, ...)
\end{verbatim}
\end{Usage}
%
\begin{Arguments}
\begin{ldescription}
\item[\code{file}] path to the data table file. For Excel files,
supports "http://", "https://", and "ftp://" URLS.

\item[\code{data.table}] (from \code{fread()}) TRUE returns a data.table.
FALSE returns a data.frame. The default for this argument can
be changed with options(datatable.fread.datatable=FALSE).

\item[\code{...}] additional arguments to read.table.
\end{ldescription}
\end{Arguments}
%
\begin{Value}
a data.frame
\end{Value}
\inputencoding{utf8}
\HeaderA{readPlainText}{readFilesText}{readPlainText}
%
\begin{Description}\relax
readFilesText
\end{Description}
%
\begin{Usage}
\begin{verbatim}
readPlainText(files, prefix = NULL, sep = "/", ...)
\end{verbatim}
\end{Usage}
%
\begin{Arguments}
\begin{ldescription}
\item[\code{files}] files basename located in the same directory or,
if prefix = NULL, list of full filenames to read

\item[\code{prefix}] common file prefix for all file names
specified in 'files'. By default, sep = "/"
\end{ldescription}
\end{Arguments}
\inputencoding{utf8}
\HeaderA{remove\_shiny\_inputs}{Clear backstage shiny observers}{remove.Rul.shiny.Rul.inputs}
%
\begin{Description}\relax
Clear server-side of a shiny module
\end{Description}
%
\begin{Usage}
\begin{verbatim}
remove_shiny_inputs(id, .input)
\end{verbatim}
\end{Usage}
%
\begin{Arguments}
\begin{ldescription}
\item[\code{id}] character. An ID string that corresponds with the ID used to call
the module's UI function.*

\item[\code{.input}] internal. Shiny server \code{input} variable passed to servers.
\end{ldescription}
\end{Arguments}
%
\begin{Details}\relax
Freely teached from a community soluce on
\Rhref{https://appsilon.com/how-to-safely-remove-a-dynamic-shiny-module/}{appsilon}.
\end{Details}
\inputencoding{utf8}
\HeaderA{renderBibliography}{renderBibliography}{renderBibliography}
%
\begin{Description}\relax
shiny-formatted render* function. Allow the user to print a .bib bibliography content. References are NOT numbered
according to possible calls from the app.
\end{Description}
%
\begin{Usage}
\begin{verbatim}
renderBibliography(bib)
\end{verbatim}
\end{Usage}
%
\begin{Arguments}
\begin{ldescription}
\item[\code{bib}] character. Path to bibliography file (.bib format).
\end{ldescription}
\end{Arguments}
\inputencoding{utf8}
\HeaderA{runMetashark}{Run MetaShARK}{runMetashark}
%
\begin{Description}\relax
Main function for launching the MetaShARK application.
\end{Description}
%
\begin{Usage}
\begin{verbatim}
runMetashark(...)
\end{verbatim}
\end{Usage}
%
\begin{Arguments}
\begin{ldescription}
\item[\code{...}] options to pass to the application, ignored if missing or mistyped.
\begin{description}

\item[wip] logical. Shows WIP parts of the app. (default to FALSE)
\item[dev] logical. Add development elements in the GUI. (default to FALSE)
\item[reactlog] logical. Use reactlog? (default to TRUE)
\item[test] logical. Recod tests? (default to FALSE)

\end{description}

\end{ldescription}
\end{Arguments}
%
\begin{Details}\relax
MetaShARK (METAdata SHiny Automated Resource \& Knowledge) is a web app
which is designed to help its user as much as possible for filling ecological
metadata. It uses the EML standard (cf. NCEAS work) to allow a full and
precise description of input datasets.

For server setup, see \Rhref{https://github.com/earnaud/MetaShARK_docker}{this git}
\end{Details}
%
\begin{Author}\relax
Elie Arnaud \Rhref{mailto:elie.arnaud@mnhn.fr}{elie.arnaud@mnhn.fr}
\end{Author}
%
\begin{Examples}
\begin{ExampleCode}
## Not run:
# run this to launch MetaShARK
runMetashark()

# End (Not run)

\end{ExampleCode}
\end{Examples}
\inputencoding{utf8}
\HeaderA{uploadUI}{uploadUI}{uploadUI}
\aliasA{upload}{uploadUI}{upload}
%
\begin{Description}\relax
UI part for the upload module. Used to build and drop
data packages to a chosen metacat.
\end{Description}
%
\begin{Usage}
\begin{verbatim}
uploadUI(id)

upload(id, main.env)
\end{verbatim}
\end{Usage}
%
\begin{Arguments}
\begin{ldescription}
\item[\code{id}] shiny module id

\item[\code{main.env}] inner global environment
\end{ldescription}
\end{Arguments}
%
\begin{Section}{Functions}
\begin{itemize}

\item{} \code{upload}: 

\end{itemize}
\end{Section}
\inputencoding{utf8}
\HeaderA{URL\_Input\_UI}{URL Input}{URL.Rul.Input.Rul.UI}
\aliasA{URL\_Input}{URL\_Input\_UI}{URL.Rul.Input}
%
\begin{Description}\relax
URL Shiny input module. Implements elementary test to get a valid URL.
\end{Description}
%
\begin{Usage}
\begin{verbatim}
URL_Input_UI(id, label = "URL", width = "100%")

URL_Input(id)
\end{verbatim}
\end{Usage}
%
\begin{Arguments}
\begin{ldescription}
\item[\code{id}] (character) shiny module inputId.

\item[\code{label}] (character) display label for the control, or NULL for no label.

\item[\code{width}] (character) the width of the input, e.g. '400px', or '100\%'; see
validateCssUnit().
\end{ldescription}
\end{Arguments}
%
\begin{Value}
(output of calling the module part)
If input is a valid URL (regex-tested + curl-tested), returns input.
Else, returns NA.
\end{Value}
%
\begin{Section}{Functions}
\begin{itemize}

\item{} \code{URL\_Input}: 

\end{itemize}
\end{Section}
\inputencoding{utf8}
\HeaderA{wipRow}{wipRow}{wipRow}
%
\begin{Description}\relax
styles an inputRow to give it WIP appearance
\end{Description}
%
\begin{Usage}
\begin{verbatim}
wipRow(...)
\end{verbatim}
\end{Usage}
%
\begin{Arguments}
\begin{ldescription}
\item[\code{...}] content to style
\end{ldescription}
\end{Arguments}
%
\begin{Value}
styled content as WIP
\end{Value}
\inputencoding{utf8}
\HeaderA{withRedStar}{withRedStar}{withRedStar}
%
\begin{Description}\relax
Add a red star at the end of the text
\end{Description}
%
\begin{Usage}
\begin{verbatim}
withRedStar(text)
\end{verbatim}
\end{Usage}
%
\begin{Arguments}
\begin{ldescription}
\item[\code{text}] the HTLM text to put before the red star
\end{ldescription}
\end{Arguments}
%
\begin{Value}
an html element
\end{Value}
%
\begin{Examples}
\begin{ExampleCode}
withRedStar("Enter your name here")

\end{ExampleCode}
\end{Examples}
\inputencoding{utf8}
\HeaderA{\Rpercent{}grep\Rpercent{}}{shorthand for base::grepl}{.Rpcent.grep.Rpcent.}
%
\begin{Description}\relax
This function is designed to be a shorthand for base::grepl, thought to be used as \code{\%in\%}.
\end{Description}
%
\begin{Usage}
\begin{verbatim}
x %grep% table
\end{verbatim}
\end{Usage}
%
\begin{Arguments}
\begin{ldescription}
\item[\code{x}] character. Pattern(s) to be matched against \code{table}.

\item[\code{table}] any unlisted data. A collection of data among which to find \code{pattern}-matching
elements.
\end{ldescription}
\end{Arguments}
\printindex{}
\end{document}
